% File name: GF2/specification/classes.tex
% handy reference sheet for names, monitors and devices classes
% Author: adh
% Date: Tue 28 May 2013 14:13

\documentclass[a4paper,11pt]{article}  % Standard document class
\usepackage[english]{babel}            % Set document language
\usepackage{fullpage}                  % Set up page for small margins etc

\usepackage{graphicx}                  % For including images in document
\usepackage{placeins}                  % Allows use of \FloatBarrier
% to avoid images or tables
% moving into next section
\usepackage{subfig}                    % For subfigures...

\usepackage{amsmath}                   % For improving maths/formula typesetting
\usepackage{tabularx}                  % Table changing package

\usepackage{algpseudocode}             % For producing algorithms/flowcharts
\usepackage{listings}                  % For including source code in document


% Provide command for scientific notation
\providecommand{\e}[1]{\ensuremath{\times10^{#1}}}
\providecommand{\degrees}{\ensuremath{^{\circ}}}

% Define title here:
\title{GF2 Classes Reference Sheet}
\author{Andrew Holt}
\date{28 May 2013}

\begin{document}

% generate title
\maketitle

\section{Names Class}

\subsection*{\texttt{vector<namestring> nametable;}}

\subsection*{\texttt{name lookup (namestring str);}}
\begin{itemize}
  \item Returns internal representation of the name given in character form.
  \item If name is not already in table, it is automatically inserted.
\end{itemize}

\subsection*{\texttt{name cvtname (namestring str);}}
\begin{itemize}
  \item Returns internal representation of the name given in character form.
  \item If name is not already in table then 'blankname' is returned.
\end{itemize}

\subsection*{\texttt{void writename (name id);}}
\begin{itemize}
  \item Prints out given name on console.
\end{itemize}

\subsection*{\texttt{void getname (name id);}}
\begin{itemize}
  \item Returns the string.
\end{itemize}

\subsection*{\texttt{int namelength (name id);}}
\begin{itemize}
  \item Returns the length (i.e. number of characters in given name).
\end{itemize}

\newpage

\section{Devices Class}

\subsection*{\texttt{void makedevice (devicekind dkind, name did, int
    variant, bool\& ok);}}
\begin{itemize}
  \item Adds device to the network.
  \item Variant for no. of inputs etc.
  \item \texttt{ok} returns \texttt{TRUE} if successful.
\end{itemize}

\subsection*{\texttt{void setswitch (name sid, asignal level, bool\& ok);}}
\begin{itemize}
  \item sets state of named switch.
  \item \texttt{ok} returns false if switch not found.
\end{itemize}

\subsection*{\texttt{void executedevices (bool\& ok);}}
\begin{itemize}
  \item Executes all devices in the network to simulate one complete
    clock cycle. 
  \item \texttt{ok} returns \texttt{FALSE} if network fails to
    stabilise.
\end{itemize}

\subsection*{\texttt{devicekind devkind (name id);}}
\begin{itemize}
  \item Returns the kind of device corresponding to the given name.
  \item \texttt{baddevice} is returned if the name is not a legal device.
\end{itemize}

\subsection*{\texttt{void writedevice (devicekind k);}}
\begin{itemize}
  \item Prints out the given device kind.
\end{itemize}

\subsection*{\texttt{void debug (bool on);}}
\begin{itemize}
  \item Used to set debugging switch.
\end{itemize}

\newpage

\section{Monitor Class}

\subsection*{\texttt{void makemonitor (name dev, name outp, bool\& ok);}}
\begin{itemize}
  \item Sets a monitor on the `outp' output of device `dev' by placing
    an entry in the monitor table.
  \item `ok' is set true if operation succeeds.
\end{itemize}

\subsection*{\texttt{void remmonitor (name dev, name outp, bool\& ok);}}
\begin{itemize}
  \item Removes the monitor set on the `outp' output of device `dev'
  \item `ok' is set true if operation succeeds.
\end{itemize}

\subsection*{\texttt{int moncount (void);}}
\begin{itemize}
  \item Returns the number of signals currently monitored.
\end{itemize}

\subsection*{\texttt{asignal getmonsignal (int n);}}
\begin{itemize}
  \item returns signal level of n'th monitor point.
\end{itemize}

\subsection*{\texttt{bool getsignaltrace (int m, int c, asignal \&s);}}
\begin{itemize}
  \item Access recorded signal trace.
  \item Returns \texttt{FALSE} if invalid monitor or cycle.
\end{itemize}

\subsection*{\texttt{void getmonname (int n, name\& dev, name\& outp);}}
\begin{itemize}
  \item Returns the name of n'th monitor
\end{itemize}

\subsection*{\texttt{string getmonprettyname (int n);}}
\begin{itemize}
  \item Returns name of device \texttt{n} in a string.
\end{itemize}


\subsection*{\texttt{void resetmonitor (void);}}
\begin{itemize}
  \item Initialises monitor memory in preparation for a new output
    sequence.
\end{itemize}

\subsection*{\texttt{void recordsignals (void);}}
\begin{itemize}
  \item Called every clock cycle to record the state of each monitored
    signal.
\end{itemize}

\subsection*{\texttt{void displaysignals (void);}}
\begin{itemize}
  \item Displays state of monitored signals.
\end{itemize}

\end{document}