% File name: Reports/First_Int_Report.tex
% First Interim Report for GF2 Software project. Details intro and
% general approach, teamwork planning, EBNF for syntax, informal
% description of semantics; error handling and example definition
% files and corresponding logic circuits
% Author: James Glanville, George Aryris and Andy Holt
% Date: Wed 15 May 2013 16:51

\documentclass[a4paper,11pt]{article}  % Standard document class
\usepackage[english]{babel}            % Set document language
\usepackage{fullpage}                  % Set up page for small margins etc

\usepackage{graphicx}                  % For including images in document
%\usepackage{placeins}                  % Allows use of \FloatBarrier
% to avoid images or tables
% moving into next section
%\usepackage{subfig}                    % For subfigures...

\usepackage{amsmath}                   % For improving maths/formula typesetting
%\usepackage{tabularx}                  % Table changing package

%\usepackage{algpseudocode}             % For producing algorithms/flowcharts

\usepackage{listings}                  % For including source code in document
\lstset{
  basicstyle = \small
}

% Provide command for scientific notation
\providecommand{\e}[1]{\ensuremath{\times10^{#1}}}
\providecommand{\degrees}{\ensuremath{^{\circ}}}

% Define title here:
\title{Project GF2: Software\\ First Interim Report\\ Software Design
  Team 1}
\author{George Ayris\\ gdwa2\\ Emmanuel \and James Glanville\\
jg597\\ Emmanuel \and Andrew Holt\\ ah635\\ Emmanuel}
\date{21 May 2013}

\begin{document}

% generate title
\maketitle

\section{Introduction}

This project requires the development of a logic simulation program,
implemented in C++. When completed, the application will read a
definition file containing a list of devices and the connections
between them. It will then graphically display the values of specified
monitor points in the circuit as the simulation is run. Some legacy
code has been provided, but lacks a scanner, parser and GUI which will
be developed in this project.

\subsection{Teamwork Planning}

The GUI development requires learning and reference to the wxWidgets
and OpenGL packages, so makes sense for one team member to focus
efforts here. The scanner and parser will require robust coding and
testing, so it was decided to combine the two as a peer programming
project to allow implementation and testing to be carried out by
different team members.

The work is to be distributed as follows: Andrew will code the GUI,
and James and George will write and test the scanner and parser. The
time frame for the development is that the software should be designed
by the end of Tuesday 21$^{\mathrm{st}}$ May; implemented and unit
tested by Tuesday 28$^{\mathrm{th}}$; allowing time for system
integration and testing by 11.00am on Friday 31$^{\mathrm{st}}$ May.

\section{Syntax Specification}

The syntax for the circuit definition files was defined and described
using the following EBNF grammar:
\begin{lstlisting}
DEFINITION = `{' DEVICES CONNECTIONS MONITORS INIT `}'

DEVICES =  `DEVICES' `{' device {device} `}'
device = devicename `=' devicetype [ `(' digit {digit} `)' ] `;'
devicename = letter {letter | digit}
devicetype = `AND' | `NAND' | `OR' | `NOR' | `XOR' | `DTYPE' | `CLK' | `SW'

CONNECTIONS = `CONNECTIONS' `{' connection {connection} `}'
connection = input `<=' output `;'
input = letter {letter|digit} [`.'letter|digit{letter|digit}]
output = letter {letter|digit} [`.'letter|digit{letter|digit}]

MONITORS = `MONITORS' `{' monitor {monitor} `}'
monitor = monitorname `<=' output `;'
monitorname = letter{letter|digit}

INIT = `INIT' `{' {init} `}'
init = devicename `=' digit{digit} `;'
\end{lstlisting}



\section{Semantics Specification}

\section{Error Handling}

\section{Example Circuits and Definition Files}

\subsection{XOR Circuit Composed of NAND Gates}

\subsection{3-bit Gray Code Counter}

\end{document}