% File name: Reports/First_Int_Report.tex
% First Interim Report for GF2 Software project. Details intro and
% general approach, teamwork planning, EBNF for syntax, informal
% description of semantics; error handling and example definition
% files and corresponding logic circuits
% Author: James Glanville, George Aryris and Andy Holt
% Date: Wed 15 May 2013 16:51

\documentclass[a4paper,11pt]{article}  % Standard document class
\usepackage[english]{babel}            % Set document language
\usepackage{fullpage}                  % Set up page for small margins etc

\usepackage{graphicx}                  % For including images in document
%\usepackage{placeins}                  % Allows use of \FloatBarrier
% to avoid images or tables
% moving into next section
%\usepackage{subfig}                    % For subfigures...

\usepackage{amsmath}                   % For improving maths/formula typesetting
%\usepackage{tabularx}                  % Table changing package

%\usepackage{algpseudocode}             % For producing algorithms/flowcharts

\usepackage{listings}                  % For including source code in document
\lstset{
  basicstyle = \small
}

% Provide command for scientific notation
\providecommand{\e}[1]{\ensuremath{\times10^{#1}}}
\providecommand{\degrees}{\ensuremath{^{\circ}}}

% Define title here:
\title{Project GF2: Software\\ First Interim Report\\ Software Design
  Team 1}
\author{George Ayris\\ gdwa2\\ Emmanuel \and James Glanville\\
jg597\\ Emmanuel \and Andrew Holt\\ ah635\\ Emmanuel}
\date{21 May 2013}

\begin{document}

% generate title
\maketitle

\section{Introduction}

This project requires the development of a logic simulation program,
implemented in C++. When completed, the application will read a
definition file containing a list of devices and the connections
between them. It will then graphically display the values of specified
monitor points in the circuit as the simulation is run. Some legacy
code has been provided, but lacks a scanner, parser and GUI which will
be developed in this project.

\subsection{Teamwork Planning}

The GUI development requires learning and reference to the wxWidgets
and OpenGL packages, so makes sense for one team member to focus
efforts here. The scanner and parser will require robust coding and
testing, so it was decided to combine the two as a peer programming
project to allow implementation and testing to be carried out by
different team members.

The work is to be distributed as follows: Andrew will code the GUI,
and James and George will write and test the scanner and parser. The
time frame for the development is that the software should be designed
by the end of Tuesday 21$^{\mathrm{st}}$ May; implemented and unit
tested by Tuesday 28$^{\mathrm{th}}$; allowing time for system
integration and testing by 11.00am on Friday 31$^{\mathrm{st}}$ May.

\section{Syntax Specification}

The syntax for the circuit definition files was defined and described
using the following EBNF grammar:
\begin{lstlisting}
DEFINITION = `{' DEVICES CONNECTIONS MONITORS INIT `}'

DEVICES =  `DEVICES' `{' device {device} `}'
device = devicename `=' devicetype [ `(' digit {digit} `)' ] `;'
devicename = letter {letter | digit}
devicetype = `AND' | `NAND' | `OR' | `NOR' | `XOR' | `DTYPE' | `CLK' | `SW'

CONNECTIONS = `CONNECTIONS' `{' connection {connection} `}'
connection = input `<=' output `;'
input = letter {letter|digit} [`.'letter|digit{letter|digit}]
output = letter {letter|digit} [`.'letter|digit{letter|digit}]

MONITORS = `MONITORS' `{' monitor {monitor} `}'
monitor = monitorname `<=' output `;'
monitorname = letter{letter|digit}

INIT = `INIT' `{' {init} `}'
init = devicename `=' digit{digit} `;'
\end{lstlisting}

\section{Semantics Specification}

The following set of semantic descriptions was created to describe the
semantics.
\begin{itemize}
  \item A logic circuit is defined in a file that opens with ``\{'' and
      closes with ``\}''.
  \item Each file has four sections: ``\texttt{DEVICES}'',
    ``\texttt{CONNECTIONS}'', ``\texttt{MONITORS}'' and
    ``\texttt{INIT}''. Each section starts with the section name
    followed by a ``\{'' and terminates with a ``\}''.
  \item Within ``\texttt{DEVICES}'' there must be at least one
    device.
 \item All names and keywords are case-insensitive,
   i.e. \texttt{GATE1.O} and \texttt{gate1.o} are the same, and must
   be unique. 
  \item A device is defined by ``\texttt{devicename =
      devicetype(N);}'', where the parentheses and parameter are only
    required for certain device types.
  \item A device name is alphanumeric and must begin with a letter.
  \item If the device has a variable number of inputs then the number
    required for the device must be specified in parentheses after the
    device type. The number of inputs specified must lie within the
    allowable range for that device type.
  \item Within ``\texttt{CONNECTIONS}'' every input (of defined
    devices) must be connected to a single device output.
  \item A connection is defined as ``\texttt{input <=
      output;}''. Where input must correspond to a defined device
    input and output to a defined device output.
  \item An input is specified by
    ``\texttt{devicename.inputidentifier}''.
  \item An output is specified by
    ``\texttt{devicename.outputidentifier}''.
  \item The clock, switch and gate outputs are accessed with
    ``\texttt{.o}''.
  \item The gate inputs are accessed with ``\texttt{.n}'' where
    \texttt{n} is 1 or 2 for XOR gates and 1-16 for all the other
    gates.
  \item The \texttt{DTYPE} inputs are accessed with ``\texttt{.d}'',
    ``\texttt{.c}'', ``\texttt{.s}'' and ``\texttt{.r}'', which
    represent \texttt{DATA}, \texttt{CLK}, \texttt{SET} and
    \texttt{RESET} (rather than \texttt{CLEAR} to avoid confusion with
    \texttt{CLK}).
  \item The \texttt{DTYPE} outputs are accessed with ``\texttt{.o}''
    and ``\texttt{.no}'' for normal output (\texttt{Q}) and inverting
    output (\texttt{QBAR}).
  \item Within ``\texttt{MONITORS}'' at least one monitor must be
    defined.
  \item A monitor is defined as ``\texttt{monitorname <= output}'',
    where outputs are specified as above.
  \item A monitorname is alphanumeric and must begin with a letter.
  \item Within ``\texttt{INIT}'' all defined switches and clocks must
    have an initialisation.
  \item Clocks are initialised with a period of \texttt{n} simulation
    cycles, where \texttt{n} is a positive integer.
  \item Switches are initialised with a state that is either 1 or 0,
    where 1 is a logic high state on the output and 0 is a logic low
    state.
  \item Initialisation is done by ``\texttt{devicename = value;}'', where
    devicename is an already defined device and value is legal for the
    device type.
  \item Comments in the definition file are defined as any data
    between an opening ``\texttt{/*}'' and a closing ``\texttt{*/}''. Comments may be
    nested.
\end{itemize}

\section{Error Handling}

\section{Example Circuits and Definition Files}

\subsection{XOR Circuit Composed of NAND Gates}

\subsection{3-bit Gray Code Counter}

\end{document}